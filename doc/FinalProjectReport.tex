\documentclass[12pt]{article}

\usepackage[utf8]{inputenc}
\usepackage[margin=1in]{geometry}
\renewcommand{\baselinestretch}{1}
\usepackage{indentfirst}

\usepackage{amsmath, amssymb}

\usepackage{hyperref}
\usepackage{cleveref}
\usepackage{graphicx}
\usepackage{float}
\graphicspath{{./figs/}}

\usepackage{natbib}
\bibliographystyle{aasjournal}

\begin{document}

\begin{center}\begin{LARGE}
\textbf{ASTR 5463 Final Project: Title}
\end{LARGE}\end{center}

\begin{center}
\textbf{Manuel Barrientos, Anthony Burrow, Adam Moss, Sarah Stangl}
\end{center}

\section{Introduction}

Trying to solve the radiative transfer equation in the non-local thermal equilibrium (NLTE) scattering problem is a hard task. In fact, if we want to use a direct approach to solve this, the amount of computer time necessary to find a solution goes over the roof. On the other hand, the 




\section{Methods}

For $i = 2, ..., N - 1$,
\begin{align*}
\alpha_i^-
&=
e_{0, i} +
\frac{e_{2, i} - (\Delta\tau_i + 2\Delta\tau_{i - 1}) e_{1, i}}
     {\Delta\tau_{i - 1} (\Delta\tau_i + \Delta\tau_{i - 1})},
\\ \beta_i^-
&=
\frac{(\Delta\tau_i + \Delta\tau_{i - 1}) e_{1, i} - e_{2, i}}
     {\Delta\tau_{i - 1} \Delta\tau_i}
\\ \gamma_i^-
&=
\frac{e_{2, i} - \Delta\tau_{i - 1} e_{1, i}}
     {\Delta\tau_i (\Delta\tau_i + \Delta\tau_{i - 1})}
\\ \alpha_i^+
&=
\frac{e_{2, i + 1} - \Delta\tau_i e_{1, i + 1}}
     {\Delta\tau_{i - 1} (\Delta\tau_i + \Delta\tau_{i - 1})}
\\ \beta_i^+
&=
\frac{(\Delta\tau_i + \Delta\tau_{i - 1}) e_{1, i + 1} - e_{2, i + 1}}
     {\Delta\tau_{i - 1} \Delta\tau_i}
\\ \gamma_i^+
&=
e_{0, i + 1} +
\frac{e_{2, i + 1} - (\Delta\tau_{i - 1} + 2\Delta\tau_i) e_{1, i + 1}}
     {\Delta\tau_i (\Delta\tau_i + \Delta\tau_{i - 1})},
\end{align*}
where
\begin{align*}
e_{0, i}
&=
1 - e^{-\Delta\tau_{i - 1}},
\\ e_{1, i}
&=
\Delta\tau_{i - 1} - e_{0, i}
\\ e_{2, i}
&=
(\Delta\tau_{i - 1})^2 - 2 e_{1, i}.
\end{align*}

\begin{align*}
\text{i}_{i - 1}^- (\mu, \nu)
&=
\Delta\text{i}_{i - 1}^- (S, \mu, \nu)
\\ &=
\gamma_{i - 1}^-,
\\ \text{i}_{i}^- (\mu, \nu)
&=
\text{i}_{i - 1}^- (\mu, \nu) e^{-\Delta\tau_{i - 1}} +
    \Delta\text{i}_{i}^- (S, \mu, \nu)
\\ &=
\gamma_{i - 1}^- e^{-\Delta\tau_{i - 1}} + \beta_{i}^-,
\\ \text{i}_{i + 1}^- (\mu, \nu)
&=
\text{i}_{i}^- (\mu, \nu) e^{-\Delta\tau_{i - 1}} +
    \Delta\text{i}_{i + 1}^- (S, \mu, \nu)
\\ &=
[\gamma_{i - 1}^- e^{-\Delta\tau_{i - 1}} + \beta_{i}^-] e^{-\Delta\tau_{i}} +
    \alpha_{i + 1}^-,
\end{align*}

\begin{align*}
\text{i}_{i + 1}^+ (\mu, \nu)
&=
\Delta\text{i}_{i + 1}^+ (S, \mu, \nu)
\\ &=
\alpha_{i + 1}^+,
\\ \text{i}_{i}^+ (\mu, \nu)
&=
\text{i}_{i + 1}^+ (\mu, \nu) e^{-\Delta\tau_{i}} +
    \Delta\text{i}_{i}^+ (S, \mu, \nu)
\\ &=
\alpha_{i + 1}^+ e^{-\Delta\tau_{i}} + \beta_{i}^+,
\\ \text{i}_{i - 1}^+ (\mu, \nu)
&=
\text{i}_{i}^+ (\mu, \nu) e^{-\Delta\tau_{i - 1}} +
    \Delta\text{i}_{i - 1}^+ (S, \mu, \nu)
\\ &=
[\alpha_{i + 1}^+ e^{-\Delta\tau_{i}} + \beta_{i}^+] e^{-\Delta\tau_{i - 1}} +
    \gamma_{i - 1}^+.
\end{align*}

\subsection{Ng Acceleration}
Once three iterations have been computed, we can use Ng Acceleration \citep{ng_1974} to compute the next S value and accelerate convergence. Ng Acceleration uses the last 4 S values to compute the next. 

The vectors $\mathbf{S^{n-1}},  \mathbf{S^{n-2}}, \mathbf{S^{n-3}}$ occupy a $\mathbf{2D}$ surface in the linear source function, $\mathbf{S}^{*}$ defined by,

\begin{equation}
    \mathbf{S^{*}} = (1-a-b)\mathbf{S^{n-1}} + a\mathbf{S^{n-2}} + b\mathbf{S^{n-3}}
\end{equation}

for constants $a,b$ such that the next iteration $\mathbf{S^{**}}$,

\begin{equation}
\begin{split}
    \mathbf{S^{**}} &= \epsilon \mathbf{B} +(1-\epsilon)\Lambda \mathbf{S^{*}}\\
    & = (1-a-b)\mathbf{S^{n}} + a\mathbf{S^{n-1}} + b\mathbf{S^{n-2}}
\end{split}
\end{equation}. 

and the magnitude-squared of the difference between the iterations,

\begin{equation}\label{eq:mini}
    | \mathbf{S^{**}} - \mathbf{S^{*}}|^{2} 
\end{equation}

is minimized. The $a,b$ values minimizing Equation \ref{eq:mini} is given by,

\begin{equation}
    a = \frac{C_{1}B_{2}-C_{2}B_{1}}{A_{1}B_{2}-A_{2}B_{1}}
\end{equation}

\begin{equation}
    b = \frac{C_{2}A_{1} - C_{1}A_{2}}{A_{1}B_{2}-A_{2}B_{1}}
\end{equation}

where,

\begin{equation}
\begin{split}
    A_{1} = Q_{1}Q_{1},\quad B_{1}=Q_{1}Q_{2},\quad C_{1}=Q_{1}Q_{3}\\
    A_{2} = Q_{2}Q_{1},\quad B_{2}=Q_{2}Q_{2},\quad C_{2}=Q_{2}Q_{3}
\end{split}
\end{equation}

and 

\begin{equation}
    \begin{split}
        & Q_{1} = \mathbf{S^{n}}-2\mathbf{S^{n-1}}+\mathbf{S^{n-2}}\\
        & Q_{2} = \mathbf{S^{n}}-\mathbf{S^{n-1}}-\mathbf{S^{n-2}}+\mathbf{S^{n-3}}\\
        & Q_{3} = \mathbf{S^{n}}-\mathbf{S^{n-1}}.
    \end{split}
\end{equation}

The new $\mathbf{S^{n+1}}$ is given by,

\begin{equation}
    \mathbf{S^{n+1}} = (1-a-b)\mathbf{S^{n}} + a\mathbf{S^{n-1}} + b\mathbf{S^{n-2}}.
\end{equation}

Before Ng Acceleration can be used again, 3 iterations must pass.


\section{Results}



\section{Conclusion}



\bibliography{FinalProjectReport}

\end{document}
